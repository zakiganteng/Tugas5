\chapter{Pendahuluan}

\section{Latar Belakang}

Pada Tahun 2015 \textit{The Organization for Economic Co-operation and Development}(OECD) merilis nilai dari \textit{Program for International Student Assessment}(PISA). Yang merupakan nilai kemampuan siswa pada matematika, membaca, dan membaca. Pada kategori matematika Indosnesia menempati peringkat 57 dari 64 negara dengan nilai 386. Sedangkan nilai rata-rata OECD untuk matematika adalah 470. Hal ini sunggu memprihatinkan oleh karena itu 

Sudoku berasal dari kata \textit{Sūji wa dokushin ni kagiru} yang berarti angkanya harus tunggal\cite{SATPy3} adalah suatu \textit{puzzle} (teka-teki) yang direpresentasikan oleh sebuah matriks (\textit{array}
dua dimensi) berukuran ${n^2 \times n^2}$  yang dibangun dari ${n^2}$ dengan submatriks (atau blok)
yang berukuran ${n \times n}$. Sudoku merupakan \textit{puzzle}
yang biasanya dimainkan oleh satu orang.  Pada
awal permainan, terdapat beberapa sel yang telah terisi yang disebut dengan pemberian
(\textit{givens}). Untuk menyelesaikan permainan ini, seorang pemain harus mengisi setiap sel yang
belum terisi dengan angka di antara 1 sampai
$n^2$ sedemikan sehingga setiap baris, setiap kolom,
dan setiap blok (submatriks berukuran $n \times n$) memuat tepat satu bilangan di antara 1 sampai $n^2$. Biasanya suatu sudoku didesain agar tepat memiliki satu kemungkinan solusi. Hal
ini juga mengakibatkan sudoku dapat diselesaikan hanya dengan mengandalkan penalaran
yang sederhana. Pengisian suatu sel dapat dilakukan dengan meninjau kemungkinan dari
isi sebuah sel.

Banyak pemain yang terjebak pada teka-teki sudoku dan tidak dapat melanjutkan permainan. Hal ini terjadi karena pemain mengisi nilai yang salah atau sudoku tidak memiliki solusi. Oleh karena itu banyak pemain yang membutuhkan pemecah sudoku yang dapat memberikan keterpenuhan dari sebuah sudoku. Sehingga pemain dapat mengetahui sebuah sudoku memiliki solusi atau tidak.

Hingga saat ini, sudah banyak penelitian yang membahas penyelesaian sudoku secara
matematis maupun komputasional. Salah satu metode yang cukup dikenal adalah penyelesaian
sudoku dengan memanfaatkan masalah keterpenuhan formula proposisional (\textit{propositional satisfiability problem}). Dengan pendekataan ini, syarat-syarat yang
harus dipenuhi oleh suatu sudoku dimodelkan dengan satu atau lebih formula logika proposisi\cite{KJ06,LO06}.
Keterpenuhan (\textit{satisfiability}) dari himpunan formula yang memodelkan syarat-syarat
ini akan menjamin bahwa suatu sudoku memiliki suatu solusi.

Pada tugas akhir ini, penulis akan membuat pemecah sudoku interaktif berbasis python. Bahasa python dipilih
karena implementasi SAT solver dalam bahasa python masih masih sedikit dibandingkan pada bahasa C dan C++\cite{SATPy1}. Aplikasi akan menggunakan pengujian kotak hitam untuk menilai kualitas aplikasi\cite{TEST1}.

\section{Perumusan Masalah}

Berdasarkan latar belakang tersebut, rumusan masalah pada tugas akhir ini
adalah bagaimana cara membuat SAT solver bebasis python untuk pemecah sudoku yang interaktif.

\section{Batasan Masalah}

Batasan pada tugas akhir ini terbatas pada:
\begin{enumerate}
	\item Penulis hanya membangun aplikasi sudoku solver berukuran ($4\times4$), ($9\times9$), ($16\times16$).
	\item Program mengeluarkan keterpenuhan dari sudoku.
	\item Program mengeluarkan solusi dari sudoku jika pengguna mengiginkan.
	\item Aplikasi akan diuji dengan metode
	pengujian kotak hitam (\textit{black box testing}).
\end{enumerate}

\section{Tujuan}

Tujuan yang ingin dicapai pada tugas akhir ini adalah untuk membuat pemecah sudoku yang interaktif.

\section{Rencana Kegiatan}

Pada pengerjaan tugas akhir ini beberapa hal yang akan saya lakukan adalah sebagai berikut:
\begin{enumerate}
	\item Analisa spesifikasi dan Kebutuhan sistem.
	\item Translasi spesifikasi sistem.
	\item Pembuatan aplikasi.
	\item Pengujian aplikasi.
	\item Analisis sistem.
	\item Penulisan laporan.
	
\end{enumerate}

\section{Jadwal Kegiatan}

Jadwal pengerjaan tugas akhir sesuai dengan alur yang telah dibuat.

\begin{table}[H]
	\caption{Jadwal Kegiatan.}
	
	\begin{center}
		\resizebox{\textwidth}{!}
		{\begin{tabular}{|c|p{5cm}|p{2cm}|p{2cm}|p{2cm}|p{2cm}|p{2cm}|p{2cm}|p{2cm}|}
				\hline 
				\multirow{2}{*}{No} & \multirow{2}{*}{Jenis Kegiatan} & \multicolumn{7}{c|}{Bulan}
				\tabularnewline
				
				\cline{3-9} 
				&  & November 2017 & Desember 2017 & Januari 2018 & Februari 2018 & Maret 2018 & April 2018 & Mei 2018
				\tabularnewline
				
				\hline 
				1 & Analisa spesifikasi dan kebutuhan sistem & \cellcolor{cyan} & \cellcolor{cyan} &  &  &  &  & \tabularnewline
				
				\hline 
				2 & Translasi spesifikasi sistem &  & \cellcolor{red} &  &  &  &  & \tabularnewline
				
				\hline 
				3 & Pembuatan aplikasi &&  
				\cellcolor{green} & \cellcolor{green} & \cellcolor{green} & \cellcolor{green} &  & \tabularnewline
				
				\hline 
				4 & Pengujian aplikasi &  &  &  &  & \cellcolor{darkgray} & \cellcolor{darkgray} & \tabularnewline
				
				\hline 
				5 & Analisis sistem &  &  &  &  &  & \cellcolor{lightgray} & \cellcolor{lightgray} \tabularnewline
				
				\hline 
				6 & Penulisan Laporan & \cellcolor{yellow} & \cellcolor{yellow} & \cellcolor{yellow} & \cellcolor{yellow} & \cellcolor{yellow} & \cellcolor{yellow} & \cellcolor{yellow} 
				\tabularnewline
				
				\hline 
		\end{tabular}}
		\par\end{center}
\end{table}