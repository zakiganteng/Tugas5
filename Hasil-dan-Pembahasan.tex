\chapter{Verifikasi Formal dan Analisisnya}

Pada bab ini dijelaskan tentang proses verifikasi formal diagram aktivitas
alur distribusi vaksin menggunakan model checker NuSMV.

\section{Langkah-Langkah Verifikasi Menggunakan NuSMV}

Pada subbab ini, penulis akan menjelaskan cara verifikasi dengan model
checker NuSMV dengan model formal dan spesifikasi yang telah dibuat
pada Bab 3. Berikut langkah verifikasi:
\begin{enumerate}
	\item Buat skrip baru dengan ekstensi .smv menggunakan text editor. Skrip
	tersebut berisi model formal yang telah ditranslasi ke NuSMV pada
	Subbab 3.3.
	\item Tambahkan spesifikasi LTL yang telah dibuat pada Subbab 3.4 dengan
	menyesuaikan aturan pada Bab 2 agar spesifikasi dapat diperiksa
	pada NuSMV.
	\item Simpan skrip ke direktori \texttt{C:/Program Files/NuSMV/<version>/bin}.
	\item Jalankan program Command Prompt. Ubah direktori ke tempat penyimpanan
	skrip yang sudah dibuat.
	\item Masukkan perintah \texttt{NuSMV.exe <namaskrip>.smv} kemudian tekan \textit{enter}.
	\item Hasil verifikasi akan terlihat ketika file NuSMV sudah selesai diproses.
	Jika spesifikasi benar maka NuSMV akan menampilkan spesifikasi is \textit{true}.
	Jika spesifikasi salah maka NuSMV akan menampilkan \textit{counter-example}
	dari spesifikasi tersebut.
\end{enumerate}

\section{Verifikasi Alur Distribusi Vaksin}

Pada Subbab ini akan dijelaskan hasil verifikasi yang dilakukan pada
model formal yang telah dibuat dan analisi dari hasil tersebut.

\subsection{Verifikasi untuk Alur Penerimaan Vaksin}

Berdasarkan spesifikasi yang telah dijelaskan pada Subbab 3.4, didapatkan
hasil verifikasi seperti berikut.

\begin{figure}[H]
	\begin{centering}
		\includegraphics[scale=0.5]{\string"C:/Users/fikri/Google Drive/Tugas_Akhir/assets/verifikasi_nusmv/PenerimaanVaksin\string"}
		\par
	\end{centering}
	
	\caption{Hasil verifikasi untuk alur penerimaan vaksin.}
\end{figure}

Pada Gambar 4.1 terlihat semua spesifikasi bernilai true. Hal itu
menunjukkan bahwa model formal telah memenuhi spesifikasi yang diberikan.

\subsection{Verifikasi untuk Alur Pemeriksaan Vaksin}

Berdasarkan spesifikasi yang telah dijelaskan pada Subbab 3.4, didapatkan
hasil verifikasi seperti berikut:

\begin{figure}[H]
	\begin{centering}
		\includegraphics[scale=0.5]{\string"C:/Users/fikri/Google Drive/Tugas_Akhir/assets/verifikasi_nusmv/PemeriksaanVaksin\string"}
		\par
	\end{centering}
	
	\caption{Hasil verifikasi untuk alur pemeriksaan vaksin.}
	
\end{figure}

Pada Gambar 4.2 terlihat semua spesifikasi bernilai true. Hal itu
menunjukkan bahwa model formal telah memenuhi spesifikasi yang diberikan.

\subsection{Verifikasi untuk Alur Cek Prioritas Pengiriman}

Berdasarkan spesifikasi yang telah dijelaskan pada Subbab 3.4, didapatkan
hasil verifikasi seperti berikut:

\begin{figure}[H]
	\begin{centering}
		\includegraphics[scale=0.5]{\string"C:/Users/fikri/Google Drive/Tugas_Akhir/assets/verifikasi_nusmv/CekPrioritasPengiriman\string"}
		\par
	\end{centering}
	
	\caption{Hasil verifikasi untuk alur cek prioritas pengiriman.}
\end{figure}

Pada Gambar 4.3 terlihat semua spesifikasi bernilai true. Hal itu
menunjukkan bahwa model formal telah memenuhi spesifikasi yang diberikan.

\subsection{Verifikasi untuk Alur Penyimpanan di PT. Bio Farma}

Berdasarkan spesifikasi yang telah dijelaskan pada Subbab 3.4, didapatkan
hasil verifikasi seperti berikut:

\begin{figure}[H]
	\begin{centering}
		\includegraphics[scale=0.5]{\string"C:/Users/fikri/Google Drive/Tugas_Akhir/assets/verifikasi_nusmv/BioFarma-01\string"}
		\par
	\end{centering}
	
	\caption{Hasil verifikasi untuk alur penyimpanan di PT. Bio Farma.}
	
\end{figure}

\begin{figure}[H]
	\begin{centering}
		\includegraphics[scale=0.5]{\string"C:/Users/fikri/Google Drive/Tugas_Akhir/assets/verifikasi_nusmv/BioFarma-02\string"}
		\par
	\end{centering}
	
	\caption{Hasil verifikasi untuk alur penyimpanan di PT. Bio Farma.}
	
\end{figure}

Pada Gambar 4.4 dan 4.5 terdapat spesifikasi yang bernilai false yaitu
\texttt{F(ac = hubungi$\_$supervisor) xor F(ac = pengiriman$\_$vaksin)}. Hal itu
menunjukkan bahwa petugas vaksin tidak akan menghubungi supervisor atau
vaksin tidak akan dikirim ke pos selanjutnya. Berdasarkan counterexample
yang diberikan oleh NuSMV, pada model formal terdapat proses yang
terus-menerus terjadi. Proses tersebut terjadi ketika \texttt{pemeriksaan\_vaksin} dan \texttt{cek\_jadwal\_ pengiriman} terus-menerus dilakukan. Data yang
didapat dari BPOM dan WHO tidak ada informasi yang menjelaskan berapa
lama waktu vaksin disimpan di ruang penyimpanan.

\subsection{Verifikasi untuk Alur Penyimpanan di Dinas Kesehatan}

Berdasarkan spesifikasi yang telah dijelaskan pada Subbab 3.4, didapatkan
hasil verifikasi seperti berikut.

\begin{figure}[H]
	\begin{centering}
		\includegraphics[scale=0.5]{\string"C:/Users/fikri/Google Drive/Tugas_Akhir/assets/verifikasi_nusmv/DinKes-01\string"}
		\par
	\end{centering}
	
	\caption{Hasil verifikasi untuk alur penyimpanan di Dinas Kesehatan.}
	
\end{figure}

\begin{figure}[H]
	\begin{centering}
		\includegraphics[scale=0.5]{\string"C:/Users/fikri/Google Drive/Tugas_Akhir/assets/verifikasi_nusmv/DinKes-02\string"}
		\par
	\end{centering}
	
	\caption{Hasil verifikasi untuk alur penyimpanan di Dinas Kesehatan.}
	
\end{figure}

Pada Gambar 4.6 dan 4.6 terdapat spesifikasi yang bernilai false yaitu \texttt{F(ac = hubungi$\_$supervisor) xor F(ac = pengirimanan$\_$vaksin)}. Hal itu menunjukkan bahwa petugas vaksin tidak akan menghubungi superviosr atau vaksin tidak akan dikirim ke pos selanjutnya. Berdasarkan \textit{counterexample} yang diberikan oleh NuSMV, pada model formal terdapat proses yang terus-menerus terjadi. Proses tersebut terjadi ketika \texttt{pemeriksaan\_vaksin} dan \texttt{cek\_jadwal\_ pengiriman} terus-menerus dilakukan. Data yang didapat dari BPOM dan WHO tidak ada informasi yang menjelaskan berapa lama waktu vaksin disimpan di ruang penyimpanan.

\subsection{Verifikasi untuk Alur Penyimpanan di Puskesmas}

Berdasarkan spesifikasi yang telah dijelaskan pada Subbab 3.4, didapatkan
hasil verifikasi seperti berikut.

\begin{figure}[H]
	\begin{centering}
		\includegraphics[scale=0.5]{\string"C:/Users/fikri/Google Drive/Tugas_Akhir/assets/verifikasi_nusmv/Puskesmas-01\string"}
		\par
	\end{centering}
	
	\caption{Hasil verifikasi untuk alur penyimpanan di Puskesmas.}
	
\end{figure}

\begin{figure}[H]
	\begin{centering}
		\includegraphics[scale=0.5]{\string"C:/Users/fikri/Google Drive/Tugas_Akhir/assets/verifikasi_nusmv/Puskesmas-02\string"}
		\par
	\end{centering}
	
	\caption{Hasil verifikasi untuk alur penyimpanan di Puskesmas.}
	
\end{figure}

Pada Gambar 4.8 dan 4.9 terdapat spesifikasi yang bernilai false yaitu
\texttt{F(ac = hubungi$\_$supervisor) xor F(ac = imunisasi)}. Hal itu
menunjukkan bahwa petugas vaksin tidak akan menghubungi supervisor atau
vaksin tidak akan dikirim ke pos selanjutnya. Berdasarkan counterexample
yang diberikan oleh NuSMV, pada model formal terdapat proses yang
terus-menerus terjadi. Proses tersebut terjadi ketika \texttt{pemeriksaan\_vaksin} dan \texttt{cek\_jadwal\_imunisasi} terus-menerus dilakukan. Data yang
didapat dari BPOM dan WHO tidak ada informasi yang menjelaskan berapa
lama waktu vaksin disimpan di ruang penyimpanan.