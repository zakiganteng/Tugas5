\chapter{Kajian Pustaka}

Pada bab ini penulis akan menjelaskan teori yang digunakan selama pengerjaan tugas akhir.

\section{Metode Formal}

Metode formal adalah teknik yang digunakan untuk memodelkan suatu
sistem yang kompleks \cite{huth2004logic}. Metode formal dapat digunakan untuk mengembangkan
sebuah aplikasi baik itu perangkat lunak atau perangkat keras. Pada
masa spesifikasi, metode formal dapat digunakan untuk memberikan gambaran
dari sistem yang akan dikembangkan sedetail yang diinginkan. Metode
formal dapat digunakan untuk memandu kegiatan pengembangan selanjutnya
dan dapat digunakan untuk memverifikasi bahwa pra-syarat untuk sistem yang
dikembangkan telah lengkap dan dapat dilanjutkan ke tingkat selanjutnya.

\section{Pengujian Kotak Hitam}

\section{Logika Proposisi}

Logika proposisi adalah bahasa formal untuk memodelkan situasi yang kita sehingga kita dapat memberi alasan tentang hal itu secara formal. Logika proposisi terdiri dari sebuah nilai kebenaran dari proposisinya.

Operator yang digunakan pada logika proposisi adalah sebagai berikut

\begin{itemize}
	\item Konjungsi(dan) pada operator ini kedua proposisi yang dihubungkan harus bernilai benar agar formula bernilai \textit{true}
	. Operator ini dilambangkan dengan $\wedge$.
	\item Disjungsi(atau) pada operator ini salah satu proposisi yang dihubungkan harus bernilai benar agar formula bernilai \textit{true}. Operator ini dilambangkan dengan$\lor$ 
	\item Negasi(tidak) pada operator ini suatu proposisi akan memiliki nilai sebaliknya dari sebuah proposisi. Operator ini dilambangkan dengan $\neg$
	\item Implikasi(jika-maka) pada operator ini digunakan untuk merepresantisakan kata \textquotedblleft jika p maka q\textquotedblright sebuah formula akan bernilai \textit{false} jika nilai p benar dan nilai q salah. Operator ini dilambangkan dengan $\to$. 
	\item Biimplikasi(jika dan hanya jika)pada operator ini suatu formula hanya akan bernilai \textit{true} jika kedua proposisinya memiliki nilai kebenaran yang sama. Operator ini dilambangkan dengan $\leftrightarrow$
	
\end{itemize}

\section{\textit{Satisfiability Problem}(SAT)}

\section{Python}

\section{\textit{Model Checker} NuSMV}

NuSMV adalah \textit{model checker} yang . . . .

\begin{algorithm} [H]
	\begin{algorithmic}[1]
		\caption{Contoh Penulisan pada NuSMV}
		\State \texttt{MODULE main}
		\State \texttt{VAR}	
		\State \texttt{$\hspace{1em}$--anggota himpunan state terdiri dari s0, s1, s2, dan s3}
		\State \texttt{$\hspace{1em}$state : {s0,s1,s2,s3};}	
		\State \texttt{ASSIGN}
		\State \texttt{$\hspace{1em}$--state dimulai dari s0}
		\State \texttt{$\hspace{1em}$init(state) := s0;}
		\State \texttt{$\hspace{1em}$--transisi antar state}
		\State \texttt{$\hspace{1em}$next(state) := }
		
		\State \texttt{$\hspace{1em}$case}
		
		\State \texttt{~~~~~~~~~~~~~~~state = s0 : {s1};}
		
		\State \texttt{~~~~~~~~~~~~~~~state = s1 : {s0,s2};}
		
		\State \texttt{~~~~~~~~~~~~~~~state = s2 : {s0,s3};}
		
		\State \texttt{~~~~~~~~~~~~~~~state = s3 : {s2};}
		
		\State \texttt{$\hspace{1em}$esac;}
		
		\State \texttt{LTLSPEC} \texttt{$\hspace{1em}$G (! (p \& q) -> F r);}	
		
		\State \texttt{LTLSPEC} \texttt{$\hspace{1em}$G ( (q \& r) -> F p);}
	\end{algorithmic}
\end{algorithm}