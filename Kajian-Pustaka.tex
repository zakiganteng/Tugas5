\chapter{Kajian Pustaka}

Pada bab ini penulis akan menjelaskan teori yang digunakan selama pengerjaan tugas akhir.

\section{Sudoku}

Sudoku berasal dari kata \textit{Sūji wa dokushin ni kagiru} yang berarti angkanya harus tunggal \cite{SATPy3}. Sudoku dipopulerkan di jepang oleh \textit{Nikoli in the paper Monthly Nikolist} pada April 1984. Sudoku adalah suatu \textit{puzzle} (teka-teki) yang direpresentasikan oleh sebuah matriks (\textit{array}
dua dimensi) berukuran ${n^2 \times n^2}$  yang dibangun dari ${n^2}$ dengan submatriks (atau blok)
yang berukuran ${n \times n}$. Sudoku merupakan \textit{puzzle}
yang biasanya dimainkan oleh satu orang.  Pada
awal permainan, terdapat beberapa sel yang telah terisi yang disebut dengan pemberian
(\textit{givens}). Untuk menyelesaikan permainan ini, seorang pemain harus mengisi setiap sel yang
belum terisi dengan angka di antara 1 sampai
$n^2$ sedemikan sehingga setiap baris, setiap kolom,
dan setiap blok (submatriks berukuran $n \times n$) memuat tepat satu bilangan di antara 1 sampai $n^2$. Biasanya suatu sudoku didesain agar tepat memiliki satu kemungkinan solusi. Hal
ini juga mengakibatkan sudoku dapat diselesaikan hanya dengan mengandalkan penalaran
yang sederhana. Pengisian suatu sel dapat dilakukan dengan meninjau kemungkinan dari
isi sebuah sel.

Sebelum sudoku modern berkembang. \textit{Puzzle latin square} terlebih dahulu berkembang yang di repretasikan oleh sebuah matriks berukuran ${n \times n}$ yang memiliki angka ${1 hingga n}$ yang harus diisi oleh pemain. untuk menyelesakain \textit{puzzle }setiap baris dan kolom pada \textit{latin square} harus memiliki nilai angka yang berbeda. \textit{Latin square} pertama kali diciptakan oleh Euler pada tahun 1783 \cite{Unk1}. Pada 19 November 1892 \textit{ Le Siècle} sebuah surat kabar di perancis mempublikasikan \textit{magic square} yaitu sebuah teka-teki  yang di repretasikan oleh sebuah matriks berukuran ${9 \times 9}$ yang memiliki dua buah submatriks berukuran ${3 \times 3}$. Tujuan dari \textit{magic square} adalah pemain mengisi baris dan kolom sehingga nilai penjumlahan setiap baris dan kolom itu sama. Sudoku modern sendiri lahir dari  Howard Garns yang dipublikasikan di \textit{Dell Magazines} pada 1979 \cite{SATPy5}.	


\section{\textit{Satisfiability Problem}(SAT)}

Masalah SAT (SAT \textit{problem}) adalah salah satu masalah penting dalam logika komputasional \cite{huth2004logic}. Dalam pengerjaannya SAT berfokus pada membuktikan sebuah klausa \textit{satisfiable} sehingga SAT berfokus pada menemukan sebuah model yang membuat klausa tersebut \textit{satisfiable}. Jika tidak ditemukan sebuah model yang membuat klausa tersebut \textit{satisfiable} maka klausa tersebut \textit{unsatisfiable}. Masalah SAT merupakan masalah NP-complete, hingga saat ini tidak
terdapat algoritma yang efisien untuk memecahkan masalah tersebut. 

\subsection{Sudoku Sebagai Masalah SAT}
Sudoku memiliki beberapa aturan contohnya untuk sudoku berukuran  $9 \times 9$ yaitu :

\begin{enumerate}
	\item Setiap baris memuat bilangan antara 1 hingga 9.
	\item Setiap kolom memuat bilangan antara 1 hingga 9.
	\item Setiap submatriks atau blok $3 \times 3$
	 memuat bilangan antara 1 hingga 9.
	\item Setiap sel memuat paling banyak satu bilangan antara 1 hingga 9.
\end{enumerate}

Dari aturan-aturan tersebut akan ditranslasikan menjadi bentuk CNF lalu digunakan pada SAT \textit{solver}. Dengan aturan yang telah ditranslasikan dalam CNF sebagai berikut:

\begin{enumerate}
	\item Setiap baris memuat bilangan antara 1 
	hingga 9 : 
	
	$\bigwedge_{r=1}^{9}$$\bigwedge_{n=1}^{9}$$\bigvee_{c=1}^{9}$$p\left(r,c,n\right)$
	
	\item Setiap kolom memuat bilangan antara 1 hingga 9 : 
	
	$\bigwedge_{c=1}^{9}$$\bigwedge_{n=1}^{9}$$\bigvee_{r=1}^{9}$$p\left(r,c,n\right)$
	
	\item Setiap submatriks atau blok $3 \times 3$
	memuat bilangan antara 1 hingga 9 : 
	
	$\bigwedge_{i=0}^{2}$$\bigwedge_{j=0}^{2}$$\bigwedge_{n=1}^{9}$$\bigvee_{r=3i+1}^{3i+3}$$\bigvee_{c=3j+1}^{3j+3}$$p\left(r,c,n\right)$
	
	\item Setiap sel memuat paling banyak satu bilangan antara 1 hingga 9 : 
	
	$\bigwedge_{r=1}^{9}$$\bigwedge_{c=1}^{9}$$\bigwedge_{n=1}^{8}$$\bigwedge_{i=n+1}^{9}$
	$\left(\neg p\left(r,c,n\right)\vee\neg p\left(r,c,i\right)\right)$
	
\end{enumerate}

Definisi dari $p\left(r,c,n\right)$ adalah sel yang memiliki perpotongan baris $r$ dan kolom $c$ dengan nilai $n$, dengan $1 \leq r,c,n \leq 9$. Sebagai contoh :

\begin{itemize}
	\item $p\left(1,3,2\right)$ berarti sel pada baris 1 dan kolom 3 memuat nilai 2
	\item $p\left(4,2,9\right)$ berarti sel pada baris 4 dan kolom 2 memuat nilai 9
\end{itemize}

Pada aturan 1 untuk menyatakan baris $r$ memuat bilangan antara 1 hingga 9, maka dapat ditulis dengan  $\bigvee_{c=1}^{9}$$p\left(r,c,n\right)$. Untuk menyatakan bahwa baris $r$ memuat semua bilangan pada $\left\{1,...,9\right\}$ maka dapat ditulis dengan $\bigwedge_{n=1}^{9}$$\bigvee_{c=1}^{9}$$p\left(r,c,n\right)$. Untuk menyatakan bahwa setiap baris memuat semua bilangan pada $\left\{1,...,9\right\}$ dapat ditulis dengan $\bigwedge_{r=1}^{9}$$\bigwedge_{n=1}^{9}$$\bigvee_{c=1}^{9}$$p\left(r,c,n\right)$

Pada aturan 2  untuk menyatakan kolom $c$ memuat bilangan antara 1 hingga 9, maka dapat ditulis dengan  $\bigvee_{r=1}^{9}$$p\left(r,c,n\right)$
Untuk menyatakan bahwa kolom $c$ memuat semua bilangan pada $\left\{1,...,9\right\}$ maka dapat ditulis dengan $\bigwedge_{n=1}^{9}$$\bigvee_{r=1}^{9}$$p\left(r,c,n\right)$. Untuk menyatakan bahwa setiap kolom memuat semua bilangan pada $\left\{1,...,9\right\}$ dapat ditulis dengan $\bigwedge_{c=1}^{9}$$\bigwedge_{n=1}^{9}$$\bigvee_{r=1}^{9}$$p\left(r,c,n\right)$
 
Pada aturan 3 karena sudoku dengan ukuran $9\times9$ memiliki sel dengan ukuran $3\times3$ maka terdapat 9 sel pada sudoku sehingga :
\begin{itemize}
	\item blok pertama akan dimulai dengan sel $\left(1,1\right)$ dan di akhiri
	pada sel $\left(3,3\right)$
	\item blok kedua akan dimulai dengan sel $\left(4,1\right)$dan di akhiri
	pada sel $\left(6,3\right)$
	\item blok ketiga akan dimulai dengan sel $\left(7,1\right)$dan di akhiri
	pada sel $\left(9,3\right)$
	\item blok keempat akan dimulai dengan sel $\left(1,4\right)$ dan di akhiri
	pada sel $\left(3,6\right)$
	\item blok kelima akan dimulai dengan sel $\left(4,4\right)$dan di akhiri
	pada sel $\left(6,6\right)$
	\item blok keenam akan dimulai dengan sel $\left(7,4\right)$dan di akhiri
	pada sel $\left(9,6\right)$
	\item blok ketujuh akan dimulai dengan sel $\left(1,7\right)$ dan di akhiri
	pada sel $\left(3,9\right)$
	\item blok kedelapan akan dimulai dengan sel $\left(4,7\right)$dan di akhiri
	pada sel $\left(6,9\right)$
	\item blok kesembilan akan dimulai dengan sel $\left(7,7\right)$dan di akhiri
	pada sel $\left(9,9\right)$
\end{itemize}

Sehingga sebuah blok memuat semua sel $\left(r,c\right)$ dengan $3i+1\leq r\leq3i+3]$
dan $3j+1\leq c\leq3j+3]$, dengan $0\leq i,j\leq2$.Oleh karena itu aturan 3 dapat ditulis dengan $\bigwedge_{i=0}^{2}$$\bigwedge_{j=0}^{2}$$\bigwedge_{n=1}^{9}$$\bigvee_{r=3i+1}^{3i+3}$$\bigvee_{c=3j+1}^{3j+3}$$p\left(r,c,n\right)$

Pada aturan 4 agar setiap sel memuat angka 1 hingga 9 dapat ditulis dengan $\bigwedge_{r=1}^{9}$$\bigwedge_{c=1}^{9}$$\bigvee_{n=1}^{9}$$p\left(r,c,n\right)$. Lalu agar setiap selnya memiliki paling banyak satu nilai maka nilai pada sel tersebut akan dibandingkan dengan nilai lainnya pada sel yang sama dan jika terdapat dua atau lebih nilai pada sel yang sama maka akan mengeluarkan nilai \textit{false} hal tersebut  dapat ditulis dengan $\bigwedge_{r=1}^{9}$$\bigwedge_{c=1}^{9}$$\bigwedge_{n=1}^{8}$$\bigwedge_{i=n+1}^{9}$
$\left(\neg p\left(r,c,n\right)\vee\neg p\left(r,c,i\right)\right)$.

\section{SAT Solver}

SAT \textit{solver} adalah sebuah aplikasi yang dibuat untuk menyelesaikan masalah SAT. Salah satu SAT \textit{solver} yang cepat adalah MiniSat. SAT \textit{solver} sendiri menerima masukan formula dalam bentuk CNF dan mengeluarkan dua buah keluaran yaitu `sat' ditambah sebuah model jika formula satisfiable, dan `unsat' jika formula unsatisfiable. Format masukan pada SAT \textit{solver} atau yang disebut DIMACS adalah sebagai berikut :

\begin{enumerate}
	\item Pertama baris komentar dapat ditulis dengan : \texttt{c <komentar>}
	\item Pertama baris komentar dapat ditulis dengan : \texttt{c <komentar>}
	\item Lalu banyak variabel dan klausa\texttt{p cnf <banyak-varian> <banyak-klausa>}
	\item Lalu klausa: 
	
	\begin{itemize}
		\item Setiap variabel di representasikan dengan sebuah integer $\geq1$.
		\item Integer dengan nilai negatif mengartikan negasi literal.
		\item Literal pada sebuah klausa dipisahkan oleh spasi.
		\item Akhir dari klausa ditulis dengan 0
	\end{itemize}
\end{enumerate}

Contoh masukan DIMACS:

Formula : $\left(\left(\ensuremath{x_{1}}\ensuremath{\vee}\ensuremath{x_{2}}\right)\ensuremath{\wedge\neg}\ensuremath{x_{3}}\right)$

Ditulis: 

\texttt{c Contoh formula satisfiable}

\texttt{p cnf 3 2} 

\texttt{1 2 0}

\texttt{-3 0}

\vspace{5mm}


Contoh keluaran SAT \textit{solver}:

Formula: $\left(\left(\ensuremath{x_{1}}\ensuremath{\vee}\ensuremath{x_{2}}\right)\ensuremath{\wedge\neg}\ensuremath{x_{3}}\right)$

\vspace{5mm}

Keluaran:

\texttt{SAT}

\texttt{1 -2 -3 0}

\vspace{5mm}

Untuk penggunaan minisat dapat ditulis dengan:
\texttt{./minisat namaFileMasukan namaFileKeluaran}

