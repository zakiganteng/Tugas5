\chapter{Kesimpulan dan Saran}

Pada bab ini dijelaskan kesimpulan dari pengerjaan tugas akhir dan saran untuk penelitian selanjutnya.

\section{Kesimpulan}

Berikut adalah kesimpulan yang didapat dari pengerjaan tugas akhir yang telah dilakukan:

\begin{enumerate}
	\item Deskripsi alur distribusi vaksin di Indonesia yang diberikan oleh PT. Bio Farma memenuhi persyaratan keamanan yang ditentukan oleh BPOM dan WHO.
	
	\item Deskripsi alur distribusi vaksin di Indonesia yang diberikan oleh PT. Bio Farma tidak cukup rinci untuk memenuhi persyaratan fungsional tambahan, yaitu persyaratan fungsional tambahan pada alur penyimpanan di PT. Bio Farma, Dinas Kesehatan, dan Puskesmas. Ini berarti terdapat kondisi pada alur penyimpanan di PT. Bio Farma, Dinas Kesehatan, dan Puskesmas yang menyebabkan vaksin tidak dikirim atau petugas vaksin menghubungi supervisor.
	
	\item Berdasarkan hasil verifikasi pada Bab 4, metode formal dapat melakukan verifikasi pada alur distribusi dan dapat menemukan kesalahan pada model yang telah dibuat.
	
\end{enumerate}

\section{Saran}

Berikut adalah saran-saran untuk penilitian selanjutnya:

\begin{enumerate}
	\item Berdasarkan hasil verifikasi pada Bab 4, diagram aktivitas belum dapat merepresentasikan secara rinci alur distribusi vaksin. Oleh karena itu, peneliti selanjutnya dapat menggunakan model lain yang lebih detail dan bervariasi, seperti dapat menangani masalah waktu.
	
	\item Pada penelitian selanjutnya, sebaiknya menggunakan bahasa formal lain agar spesifikasi formal yang akan diverifikasi dapat lebih detail.
\end{enumerate}